
% This LaTeX was auto-generated from MATLAB code.
% To make changes, update the MATLAB code and republish this document.

\documentclass{article}
\usepackage{graphicx}
\usepackage{color}

\sloppy
\definecolor{lightgray}{gray}{0.5}
\setlength{\parindent}{0pt}

\begin{document}

    
    
\subsection*{Contents}

\begin{itemize}
\setlength{\itemsep}{-1ex}
   \item Homework 4                                    Student:      Daniel Clark
   \item Problem
   \item Displacement Equation
   \item Constants
   \item Always Normal Variables
   \item Assumed Normal Variable
   \item Problem 1
   \item MCS
   \item Solution
   \item Problem 2
   \item Solution
   \item Problem 3
   \item Solution
   \item Problem 4
   \item log Normal
   \item MCS
   \item Solution
\end{itemize}


\subsection*{Homework 4                                    Student:      Daniel Clark}

\begin{verbatim}
clear all                                      % Instructor: Dr. Ha-Rok Bae
close all                                      % Class: ME 7060 Spring 2016
clc
format shorte, warning('off')
\end{verbatim}


\subsection*{Problem}

\begin{par}
The failure event is defined as $delta_{max} > 2.0$. Answer the questions below.
\end{par} \vspace{1em}


\subsection*{Displacement Equation}

\begin{verbatim}
delta_max = @(P,L,E,I,w) (P.*L.^3)./ (48.*E.*I) + (5*w.*L.^4)./(385.*E.*I);
\end{verbatim}


\subsection*{Constants}

\begin{verbatim}
L = 30*12;                  % was ft now in
I = 1.33*10^3;              % in^4
\end{verbatim}


\subsection*{Always Normal Variables}

\begin{verbatim}
mu_P = 50*10^3;             % was kip now lbs
mu_E = 29*10^6;             % lb/in^2
sigma_P = 10*10^3;          % was kip now lbs
sigma_E = 2*10^6;           % lb/in^2
\end{verbatim}


\subsection*{Assumed Normal Variable}

\begin{verbatim}
mu_W = (1/12)*1000;                % was kip/ft now lbs/in
sigma_W = (0.1/12)*1000;           % was kip/ft now lbs/in
\end{verbatim}


\subsection*{Problem 1}

\begin{par}
Implement your MVFOSM matlab codes and estimate the probability of failure event. (For this question, assume w is a normal variable with a mean of 1 kip/ft and a standard deviation of 0.1 kip/ft.)
\end{par} \vspace{1em}


\subsection*{MCS}

\begin{verbatim}
Number_of_runs = 10^6;
P_v = mu_P + sigma_P.*icdf('normal',rand(Number_of_runs,1),0,1);
E_v = mu_E + sigma_E.*icdf('normal',rand(Number_of_runs,1),0,1);
W_v = mu_W + sigma_W.*icdf('normal',rand(Number_of_runs,1),0,1);

X = [P_v, E_v, W_v];
[responseVector,~] = LSB(X);
pf_true_MCS = sum(responseVector < 0) / Number_of_runs
\end{verbatim}

        \color{lightgray} \begin{verbatim}
pf_true_MCS =

   1.7824e-01

\end{verbatim} \color{black}
    

\subsection*{Solution}

\begin{verbatim}
X1 = [mu_P, mu_E, mu_W];
[g, gDelta] = LSB(X1);
gtil = g + gDelta(1) * (X1(1) - mu_P) + gDelta(2) * (X1(2) - mu_E) ...
    + gDelta(3) * (X1(3) - mu_W);
sigma_g = sqrt( (gDelta(1) * sigma_P)^2 + (gDelta(2) * sigma_E)^2 ...
    + (gDelta(3) * sigma_W)^2 );
beta = gtil/sigma_g;
pf = 1 - cdf('normal',beta,0,1)

percentError_from_MCS = ( (pf_true_MCS - pf) / pf_true_MCS ) * 100
\end{verbatim}

        \color{lightgray} \begin{verbatim}
pf =

   1.7107e-01


percentError_from_MCS =

   4.0227e+00

\end{verbatim} \color{black}
    

\subsection*{Problem 2}

\begin{par}
Build your Hasofer Lind (HL) Method and estimate the failure probability. (For this question, assume w is a normal variable with a mean of 1 kip/ft and a standard deviation of 0.1 kip/ft.)
\end{par} \vspace{1em}


\subsection*{Solution}

\begin{par}
Step A Iteration 1 Set the mean value point as an initial design point and set the required $\beta$ convergence tolerance to $\epsilon_r = 0.001$ Compute the limit-state function value and gradients at the mean value point:
\end{par} \vspace{1em}
\begin{verbatim}
betaHistory = [];
X1 = [mu_P, mu_E, mu_W];
[g, gDelta] = LSB(X1);
gtil = g; % For the first point
sigma_g = sqrt( (gDelta(1) * sigma_P)^2 + (gDelta(2) * sigma_E)^2 ...
    + (gDelta(3) * sigma_W)^2 );

% Step B Compute the initial beta using the mean-value method and its
% direction cosine
beta = gtil/sigma_g;
betaHistory = [betaHistory; beta];

alphaP = -gDelta(1)*sigma_P / sigma_g;
alphaE = -gDelta(2)*sigma_E / sigma_g;
alphaW = -gDelta(3)*sigma_W / sigma_g;

% Step C Compute a new design point 2 X from Equation
Xnew = [mu_P+beta*sigma_P*alphaP, mu_E+beta*sigma_E*alphaE,...
    mu_W+beta*sigma_W*alphaW]

U_P = ( Xnew(1) - mu_P ) / sigma_P;
U_E = ( Xnew(2) - mu_E ) / sigma_E;
U_W = ( Xnew(3) - mu_W ) / sigma_W;
beta_previous = 1000; % helps with the loop

while (beta_previous - beta) > 0.00000001
% Iteration 2: Step A
beta_previous = beta;
[g, gDelta] = LSB(Xnew);

gtil = g - (gDelta(1)*U_P*sigma_P + gDelta(2)*U_E*sigma_E  ...
    + gDelta(3)*U_W*sigma_W);
sigma_g = sqrt( (gDelta(1) * sigma_P)^2 + (gDelta(2) * sigma_E)^2 ...
    + (gDelta(3) * sigma_W)^2 );
beta = gtil/sigma_g;

alphaP = -gDelta(1)*sigma_P / sigma_g;
alphaE = -gDelta(2)*sigma_E / sigma_g;
alphaW = -gDelta(3)*sigma_W / sigma_g;

Xnew = [mu_P+beta*sigma_P*alphaP, mu_E+beta*sigma_E*alphaE,...
    mu_W+beta*sigma_W*alphaW];

U_P = ( Xnew(1) - mu_P ) / sigma_P;
U_E = ( Xnew(2) - mu_E ) / sigma_E;
U_W = ( Xnew(3) - mu_W ) / sigma_W;
betaHistory = [betaHistory; beta];

end

betaHistory
pf = 1 - cdf('normal',beta,0,1)
percentError_from_MCS = ( (pf_true_MCS - pf) / pf_true_MCS ) * 100
\end{verbatim}

        \color{lightgray} \begin{verbatim}
Xnew =

   5.8465e+04   2.8198e+07   8.4652e+01


betaHistory =

   9.4995e-01
   9.2277e-01
   9.2284e-01


pf =

   1.7805e-01


percentError_from_MCS =

   1.0908e-01

\end{verbatim} \color{black}
    

\subsection*{Problem 3}

\begin{par}
Build a quadratic regression model of LSF with the 3-level samples within the ranges of �3? and make an estimation of the failure probability by running MCS with 1 million samples generated from the regression model. (For this question, assume w is a normal variable with a mean of 1 kip/ft and a standard deviation of 0.1 kip/ft.)
\end{par} \vspace{1em}


\subsection*{Solution}

\begin{verbatim}
Design = fullfact([3 3 3])-2;
Design = Design*3;
Design(:,1) = Design(:,1)*sigma_P; Design(:,2) = Design(:,2)*sigma_E;
Design(:,3) = Design(:,3)*sigma_W;
Means = [mu_P*ones(length(Design),1), mu_E*ones(length(Design),1), ...
    mu_W*ones(length(Design),1)];
Samples = Means + Design;
[SampleResponse, ~] = LSB(Samples);
Tables = fitlm(Samples,SampleResponse,'purequadratic');
b = table2array(Tables.Coefficients(:,1));
LSF = @(P, E, W) b(1) + b(2).*P + b(3).*E + b(4).*W + ...
    b(5).*P.^2 + b(6).*E.^2 + b(7).*W.^2;

Number_of_runs = 10^6;
P_v = mu_P + sigma_P.*icdf('normal',rand(Number_of_runs,1),0,1);
E_v = mu_E + sigma_E.*icdf('normal',rand(Number_of_runs,1),0,1);
W_v = mu_W + sigma_W.*icdf('normal',rand(Number_of_runs,1),0,1);

responseVector = LSF(P_v, E_v, W_v);
pf = sum(responseVector < 0) / Number_of_runs
percentError_from_MCS = ( (pf_true_MCS - pf) / pf_true_MCS ) * 100
\end{verbatim}

        \color{lightgray} \begin{verbatim}
pf =

   2.1930e-01


percentError_from_MCS =

  -2.3037e+01

\end{verbatim} \color{black}
    

\subsection*{Problem 4}

\begin{par}
Build your Hasofer Lind � Rackwitz Fiessler (HL-RF) Method and estimate the failure probability. (For this question, assume w is following a lognormal distribution with a mean of 1 kip/ft and a standard deviation of 0.1 kip/ft.)
\end{par} \vspace{1em}


\subsection*{log Normal}

\begin{verbatim}
mu_w = 1*10^3/12;                   % was kip/ft now lbs/in
sigma_w = 0.1*10^3/12;              % was kip/ft now lbs/in
mu_log_w = log((mu_w^2)/sqrt(sigma_w^2+mu_w^2));
sigma_log_w = sqrt(log((sigma_w^2/(mu_w^2))+1));
CDF = @(w) cdf('lognormal', w, mu_log_w, sigma_log_w);
PDF = @(w) pdf('lognormal', w, mu_log_w, sigma_log_w);
\end{verbatim}


\subsection*{MCS}

\begin{verbatim}
Number_of_runs = 10^6;
P_v = mu_P + sigma_P.*icdf('normal',rand(Number_of_runs,1),0,1);
E_v = mu_E + sigma_E.*icdf('normal',rand(Number_of_runs,1),0,1);
W_v = exp(mu_log_w + sigma_log_w.*icdf('normal',rand(Number_of_runs,1),0,1));

X = [P_v, E_v, W_v];
[responseVector,~] = LSB(X);
pf_true_MCS = sum(responseVector < 0) / Number_of_runs
\end{verbatim}

        \color{lightgray} \begin{verbatim}
pf_true_MCS =

   1.7806e-01

\end{verbatim} \color{black}
    

\subsection*{Solution}

\begin{verbatim}
betaHistory = [];
sigma_W = pdf('norm',icdf('norm',CDF(mu_W),0,1),0,1)/PDF(mu_W);
mu_W = mu_W-icdf('norm',CDF(mu_W),0,1)*sigma_W;
X1 = [mu_P, mu_E, mu_W];

[g, gDelta] = LSB(X1);
gtil = g; % For the first point
sigma_g = sqrt( (gDelta(1) * sigma_P)^2 + (gDelta(2) * sigma_E)^2 ...
    + (gDelta(3) * sigma_W)^2 );

% Step B Compute the initial beta using the mean-value method and its
% direction cosine
beta = gtil/sigma_g;
betaHistory = [betaHistory; beta];

alphaP = -gDelta(1)*sigma_P / sigma_g;
alphaE = -gDelta(2)*sigma_E / sigma_g;
alphaW = -gDelta(3)*sigma_W / sigma_g;

% Step C Compute a new design point 2 X from Equation
Xnew = [mu_P+beta*sigma_P*alphaP, mu_E+beta*sigma_E*alphaE,...
    mu_W+beta*sigma_W*alphaW];

U_P = ( Xnew(1) - mu_P ) / sigma_P;
U_E = ( Xnew(2) - mu_E ) / sigma_E;
U_W = ( Xnew(3) - mu_W ) / sigma_W;
beta_previous = 1000; % helps with the loop

while (beta_previous - beta) > 0.00000001
% Iteration 2: Step A
beta_previous = beta;
sigma_W = pdf('norm',icdf('norm',CDF(mu_W),0,1),0,1)/PDF(mu_W);
mu_W = mu_W-icdf('norm',CDF(mu_W),0,1)*sigma_W;

[g, gDelta] = LSB(Xnew);

gtil = g - (gDelta(1)*U_P*sigma_P + gDelta(2)*U_E*sigma_E  ...
    + gDelta(3)*U_W*sigma_W);
sigma_g = sqrt( (gDelta(1) * sigma_P)^2 + (gDelta(2) * sigma_E)^2 ...
    + (gDelta(3) * sigma_W)^2 );
beta = gtil/sigma_g;

alphaP = -gDelta(1)*sigma_P / sigma_g;
alphaE = -gDelta(2)*sigma_E / sigma_g;
alphaW = -gDelta(3)*sigma_W / sigma_g;

Xnew = [mu_P+beta*sigma_P*alphaP, mu_E+beta*sigma_E*alphaE,...
    mu_W+beta*sigma_W*alphaW];

U_P = ( Xnew(1) - mu_P ) / sigma_P;
U_E = ( Xnew(2) - mu_E ) / sigma_E;
U_W = ( Xnew(3) - mu_W ) / sigma_W;
betaHistory = [betaHistory; beta];
end
betaHistory
pf = 1 - cdf('normal',beta,0,1)
percentError_from_MCS = ( (pf_true_MCS - pf) / pf_true_MCS ) * 100
\end{verbatim}

        \color{lightgray} \begin{verbatim}
betaHistory =

   9.5853e-01
   9.3088e-01
   9.3105e-01


pf =

   1.7591e-01


percentError_from_MCS =

   1.2064e+00

\end{verbatim} \color{black}
    


\end{document}
    
